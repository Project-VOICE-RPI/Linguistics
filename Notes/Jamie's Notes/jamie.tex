\documentclass{article}
\usepackage[utf8]{inputenc}

\author{Julia Rahmeier}
\title{Project VOICE: Notes}

\begin{document}
	
	\maketitle
	
	\section{10/08/2024}
	- Familiarizing myself with current AAC (augmentative and alternative communication) being used with people with PIMD
    - Possible uses of virtual reality for those with general communication disabilities
    - Questions: 
        - What do the current practices speech pathologists use for people with PIMD (technological or otherwise) lack?

	
	\section{10/15/2024}
	Currently looking into existing assistive communication devices. \\
	https://files.eric.ed.gov/fulltext/EJ814395.pdf \\
	
	Letterboards \\
	- Simple, accessible, but heavily questioned since the user needs help from an interpreter who can bias or misread their answers and intent. \\
	- HoloBoard, use of AR (https://dl.acm.org/doi/10.1145/3613904.3642626)\\
    - Use of eye-tracking instead of a need for fine-tuned gestures \\
	- Sensory and privacy issues \\
	\\
	
	Sign Language (Total Communication) \\
	- Useful for a certain population (functionally mute: has intact language abilities and literacy, but can’t speak:  3/10 of the autistic population in 2007). \\
	- Signs are more iconic and require less symbolic processing than regular speech. \\
	- Useful for people with mostly auditory-vocal processing difficulties, but requires finely-tuned movement, so it’s not very helpful for people with visual-motor difficulty. \\
	\\
	
	Visual-Graphic Symbols (pictograms, photographs) \\
	- Account for individuals who lack the cognitive abilities essential to signing and speaking (poor imitation skills and motor functioning disorders) \\
	\\
	
	Questions/future considerations:\\
	- Importance of imitation skills for language development.\\
	- How widespread visual-motor and auditory-vocal processing difficulties are between people with ASD and general PIMD (profound intellectual and multiple disabilities) and to what levels.\\
	- Look at naturalistic strategies for language learning for those populations. It might be useful in designing or analyzing a product.\\
	
	

\end{document}