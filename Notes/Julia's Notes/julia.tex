\documentclass{article}
\usepackage[utf8]{inputenc}

\author{Julia Rahmeier}
\title{Project VOICE: Notes}

\begin{document}
	
	\maketitle
	
	\section{10/08/2024}
	Currently looking at measures of cognition in nonspeaking individuals and assistive methods. \\
	
	https://srcd.onlinelibrary.wiley.com/doi/full/10.1111/j.1467-8624.2011.01619.x \\
	
	Children with ASD process audiovisual information in a significantly different way, which might contribute to their language and communication impairment.
	Differences in attention during social interactions (looking less at the speaker) may have even affected the development of language perception. \\\\
	Questions: \\
	The main groups included in nonspeaking research are people with ASD or Cerebral Palsy. Should we include both in our research or focus on only one?
	
	\section{10/15/2024}
	Currently looking into existing assistive communication devices. \\
	https://files.eric.ed.gov/fulltext/EJ814395.pdf \\
	
	Letterboards \\
	- Simple, accessible, but heavily questioned since the user needs help from an interpreter who can bias or misread their answers and intent. \\
	- HoloBoard, use of AR (https://dl.acm.org/doi/10.1145/3613904.3642626)\\
    - Use of eye-tracking instead of a need for fine-tuned gestures \\
	- Sensory and privacy issues \\
	\\
	
	Sign Language (Total Communication) \\
	- Useful for a certain population (functionally mute: has intact language abilities and literacy, but can’t speak:  3/10 of the autistic population in 2007). \\
	- Signs are more iconic and require less symbolic processing than regular speech. \\
	- Useful for people with mostly auditory-vocal processing difficulties, but requires finely-tuned movement, so it’s not very helpful for people with visual-motor difficulty. \\
	\\
	
	Visual-Graphic Symbols (pictograms, photographs) \\
	- Account for individuals who lack the cognitive abilities essential to signing and speaking (poor imitation skills and motor functioning disorders) \\
	\\
	
	Questions/future considerations:\\
	- Importance of imitation skills for language development.\\
	- How widespread visual-motor and auditory-vocal processing difficulties are between people with ASD and general PIMD (profound intellectual and multiple disabilities) and to what levels.\\
	- Look at naturalistic strategies for language learning for those populations. It might be useful in designing or analyzing a product.\\
	
	

\end{document}